\documentclass{sig-alternate}

%-- Begin patch area for accents in 'Author Block' area - may be needed by some authors / but not all
\DeclareFixedFont{\auacc}{OT1}{phv}{m}{n}{12}   % Needed for "Author Block" accents - Patch by Gerry 3/21/07
\DeclareFixedFont{\afacc}{OT1}{phv}{m}{n}{10}   % Needed for "Author Block" accents in the affiliation/address line - Patch by Gerry 3/21/07
%--
\begin{document}
%
% --- Author Metadata here ---
%\conferenceinfo{WOODSTOCK}{'97 El Paso, Texas USA}
\CopyrightYear{2007}
%\crdata{0-12345-67-8/90/01}  % Allows default copyright data (0-89791-88-6/97/05) to be over-ridden - IF NEED BE.
% --- End of Author Metadata ---

\title{An Algorithm For Discovering Similar Blogs
\titlenote{Copyright is held by the authors}}
%\subtitle{Foo Foo}
%\titlenote{A full version of this paper is available as
%\textit{Author's Guide to Preparing ACM SIG Proceedings Using
%\LaTeX$2_\epsilon$\ and BibTeX} at
%\texttt{www.acm.org/eaddress.htm}}}
%

\numberofauthors{3}
\author{
% 1st. author
\alignauthor
Eran Chinthaka \\
       \affaddr{Indiana University Computer Science Department}\\
       \affaddr{150 S Woodlawn Avenue}\\
       \affaddr{Bloomington, IN 47405}\\
       \email{echintha@cs.indiana.edu}
% 2nd. author
\alignauthor
Michael D. Conover \\
       \affaddr{Indiana University School of Informatics}\\
       \affaddr{901 E. 10th St.}\\
       \affaddr{Bloomington, IN 47408}\\
       \email{midconov@indiana.edu}
% 3rd. author
\alignauthor Michel A. Salim \\
       \affaddr{Indiana University Computer Science Department}\\
       \affaddr{150 S Woodlawn Avenue}\\
       \affaddr{Bloomington, IN 47405}\\
       \email{msalim@cs.indiana.edu}
}

\date{26 April 2007}

\maketitle
\begin{abstract}

%We hypothesize that blogs form a web of communities; i.e. that the
%link structure exposes semantic similarity. We have developed a
%framework that rank blogs based on their similarity to a query.
%
% From Mike
Every day 175,000 new blogs are created, added at a rate of two per
second to the sixty million-plus online journals already chronicling
everything from politics, knitting, and finance to the latest advances
in the sciences and technology~\cite{Disc07}. With this explosive growth come
many unique challenges, and chief among them is the identification of
information most relevant to each unique information consumer.  To
address this problem we have created a system which harnesses the
social network-like topology of the blogosphere to recommend relevant
blogs to users based on sites they already know and enjoy.
%%%

\end{abstract}

% A category with the (minimum) three required fields
\category{H.4}{Information Systems Applications}{Miscellaneous}
%A category including the fourth, optional field follows...
\category{D.2.8}{Software Engineering}{Metrics}[complexity measures, performance measures]

\terms{blogosphere, blogroll, FOAF}

\keywords{ACM proceedings, \LaTeX, text tagging}

\section{Introduction}

%Past studies have shown that the World Wide Web is organized into web
%communities, and that semantically similar sites can be identified
%based on their link structures.

%\textbf{[TODO: Mike, have a look at this]} -- We hypothesize that a
%similar organizational structure is present in the blogosphere: that
%blogs are more likely to link to blogs of a similar topic. As such,
%algorithms that have been shown to work on the web in general should
%also work, but we also develop a novel algorithm that is more suited
%to the link patterns observed in blogs.
%
%From Mike
The blogosphere is a highly social place; an analysis of its
hyperlink structure demonstrates the emergence of tightly-clustered
topical communities of bloggers with many participants engaged in
interactive discussions of current events. It is this social property
of the blogosphere that motivates the
``friend-of-a-friend''-style recommendation system we
have proposed. Intuitively, the more friends two strangers have in
common, the more likely it is that they share common tastes and
interests.  Likewise, because the blogosphere exhibits many properties
of a social network, it is reasonable to posit that the more neighbors
two blogs have in common, the more they share some common appeal.
%%%

\section{Related Work}

Gibson, Kleinberg and Raghavan~\cite{gibson1998iwc} studied the link
structure of Internet sites and how they can be used to infer the
existence of communities of related sites.

Dean and Henzinger~\cite{dean1999frp} described two algorithms (one
derived from HITS) that use link information to identify related web
pages.

\section{Graph of the Blogosphere}
In addition to the regular hyperlinks present in posts, a common
feature of many blogs is the ``blogroll'', a set of
links to other sites that the author has identified as interesting and
enjoyable.  It is because of this characteristic of blogroll links
that led us to choose them as the primary source for network data in
the development of our recommendation system.

To populate our dataset we utilized information provided by the site
\texttt{blogrolling.com}, which monitors the inbound blogroll
connections to thousands of sites.  We seeded our data collection
efforts with the URLs of the 500 most popular blogs, as identified by
the blog-monitoring site technorati.com.  The inbound links to each of
these sites according to blogrolling.com were recorded, and the source
URLs were added to the collection queue.  This process was allowed to
continue until we had identified 467,477 links between 50,448 blogs.
%{CHANGE THIS WHEN DATA IS CLEANED}.

\section{Recommendation}
Our FOAF technique allows a user to find blogs that are similar to one
with which he is already familiar.  To accomplish this the algorithm
must first identify a set of candidate blogs and then rank the
members of that set according to a scoring measure.

For any given blog it is be trivial for a user to identify the sites
listed in the blogroll as candidates of potential interest.  Ranking
these sites in terms of relevance may be more difficult due to the
sheer size of many blogrolls, however this task is still tractable if
the user visits each site and makes a decision about relevance in that
manner.  To extend this process further, and rank the sites listed in
the blogrolls of each one of the initial set of sites would be
prohibitively tedious.  We seek to approximate this process
algorithmically instead.

\subsection{Identifying Candidates}
Our system mimics the user's decision making behavior in the following
way.  For the original blog of interest, $B$, the FOAF algorithm
populates a network with directed edges from a node representing $B$ to
nodes representing all of the blogs listed in $B$'s blogroll.  This set
of sinks is called $D_1$.  This process is then repeated for each of the
blog-nodes in the $D_1$ set, with the resulting sinks becoming memebers
of another set, $D_2$.  It should be noted that some of the nodes in $D_1$
may be linked to by other nodes in $D_1$, and thus have membership in
both $D_1$ and $D_2$.  Through this process our algorithm identifies all of
the ``friends-of-a-friend'' for the original blog, $B$.

\subsection{Ranking Candidates}
We experimented with three measures used for scoring blogs, all based
on the number of mutual neighbors, or friends, that a specific member
of $D_2$, called $D'$, shares with the original blog $B$.

\subsection{Raw Score}
The first  measure with which we experimented  simply scores according
to the  number of mutual  neighbors between $D'$  and $B$.  This  score is
normalized to be between zero and  one by dividing by the total number
of possible neighbors.

\begin{equation}
RS(D') = \frac{\lvert Neighbors \rvert}{\lvert Nodes in D_1 \rvert}
\end{equation}

One problem with this approach is that extremely popular blogs with
very high in-degree will naturally tend to have more neighbors in
common with any given node simply due to their high levels of
connectivity.  As a result, this measure leads to an
over-representation of highly popular sites in the result set.

\subsection{Community-Weighted Measure}
To address the problem of over-representation of the most popular
nodes we decided to try an approach that weighted the original raw
score for a given node by the degree to which it was a member of the
community surrounding the original node B.  For each node this is
approximated by dividing the number of in-links from members $D_1$ and $D_2$
by the total in-degree of that node.

\begin{equation}
S2(D') = RS(D') \times \frac{[Inlinks from D_1 + D_2]}{[ Global In-Degree]}
\end{equation}

While this did effectively eliminate the over-representation of
globally popular nodes from the result set, it also penalized blogs
that were legitimately related to blog B but had a broad appeal.  For
example, a photography blog that also covered technology-related
issues would appeal to members of both photography and technology
communities, and thus the relative in-degree from members of either
community exclusively is diminished.

%\section{Algorithms}
%\subsection{Secret Sauce}
%This algorithm is based on Friend-of-a-Friend (FOAF). The rationale is
%that a blog is likely to cite other blogs (that we call the \emph{D_1}
%set, for distance=1) that share at least a single topic of common
%interest. The blogs that these blogs cite (\emph{D_2}) should thus also
%include those on that shared topic, and the more paths there are to a
%\emph{D_2} blog (through any blog in \emph{D_1}), the more likely this
%blog is to be relevant, since more of the source blog's friend agrees
%on it.

%\subsection{HITS}

%\subsection{HITS with Co-citation and Co-reference}
%This uses the standard HITS implementation, but the initial subgraph
%is enriched with sites that co-refer the same targets and sites that
%are co-cited (a parent site links to both the source site and this
%site)

%\section{Implementation}
%\subsection{Graph Framework}
%We build our graph framework on top of the open-source Java Unified
%Network/Graph Framework, adding

%\subsection{Web Application}
%We have written a web interface 

%\section{Results}
%\section[Discussion}
\section{Conclusions}

We have developed an algorithm for ranking blogs in the link
neighborhood of a query blog such that the resulting ordering
approximates their semantic similarity to the query.

\section{Future Work}
We plan to do a double-blind user study, aimed at answering the
following questions:
\begin{enumerate}
\item How well does link structure predict semantic structure,
  regardless of the specific algorithm
\item How well does our algorithm fare against similar algorithms,
  e.g. those derived from HITS

\end{enumerate}

%ACKNOWLEDGMENTS are optional
\section{Acknowledgments}
This section is optional; it is a location for you
to acknowledge grants, funding, editing assistance and
what have you.  In the present case, for example, the
authors would like to thank Gerald Murray of ACM for
his help in codifying this \textit{Author's Guide}
and the \textbf{.cls} and \textbf{.tex} files that it describes.

%
% The following two commands are all you need in the
% initial runs of your .tex file to
% produce the bibliography for the citations in your paper.
\bibliographystyle{abbrv}
\bibliography{web-mining}
% You must have a proper ".bib" file
%  and remember to run:
% latex bibtex latex latex
% to resolve all references
%
% ACM needs 'a single self-contained file'!
%

\subsection{References}
Generated by bibtex from your ~.bib file.  Run latex,
then bibtex, then latex twice (to resolve references)
to create the ~.bbl file.  Insert that ~.bbl file into
the .tex source file and comment out
the command \texttt{{\char'134}thebibliography}.
% This next section command marks the start of
% Appendix B, and does not continue the present hierarchy
\section{More Help for the Hardy}
The sig-alternate.cls file itself is chock-full of succinct
and helpful comments.  If you consider yourself a moderately
experienced to expert user of \LaTeX, you may find reading
it useful but please remember not to change it.
%\balancecolumns % GM MARCH 2007
% That's all folks!
\end{document}
