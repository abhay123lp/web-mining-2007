\documentclass{sig-alternate}

%-- Begin patch area for accents in 'Author Block' area - may be needed by some authors / but not all
\DeclareFixedFont{\auacc}{OT1}{phv}{m}{n}{12}   % Needed for "Author Block" accents - Patch by Gerry 3/21/07
\DeclareFixedFont{\afacc}{OT1}{phv}{m}{n}{10}   % Needed for "Author Block" accents in the affiliation/address line - Patch by Gerry 3/21/07
%--
\begin{document}
%
% --- Author Metadata here ---
%\conferenceinfo{WOODSTOCK}{'97 El Paso, Texas USA}
\CopyrightYear{2007}
%\crdata{0-12345-67-8/90/01}  % Allows default copyright data (0-89791-88-6/97/05) to be over-ridden - IF NEED BE.
% --- End of Author Metadata ---

\title{An Algorithm For Discovering Similar Blogs
\titlenote{Copyright is held by the authors}}
%\subtitle{Foo Foo}
%\titlenote{A full version of this paper is available as
%\textit{Author's Guide to Preparing ACM SIG Proceedings Using
%\LaTeX$2_\epsilon$\ and BibTeX} at
%\texttt{www.acm.org/eaddress.htm}}}
%

\numberofauthors{3}
\author{
% 1st. author
\alignauthor
Eran Chinthaka \\
       \affaddr{Indiana University Computer Science Department}\\
       \affaddr{150 S Woodlawn Avenue}\\
       \affaddr{Bloomington, IN 47405}\\
       \email{echintha@cs.indiana.edu}
% 2nd. author
\alignauthor
Michael D. Conover \\
       \affaddr{Indiana University School of Informatics}\\
       \affaddr{901 E. 10th St.}\\
       \affaddr{Bloomington, IN 47408}\\
       \email{midconov@indiana.edu}
% 3rd. author
\alignauthor Michel A. Salim \\
       \affaddr{Indiana University Computer Science Department}\\
       \affaddr{150 S Woodlawn Avenue}\\
       \affaddr{Bloomington, IN 47405}\\
       \email{msalim@cs.indiana.edu}
}

\date{26 April 2007}

\maketitle
\begin{abstract}

We hypothesize that blogs form a web of communities; i.e. that the
link structure exposes semantic similarity. We have developed a
framework that rank blogs based on their similarity to a query.


\end{abstract}

% A category with the (minimum) three required fields
\category{H.4}{Information Systems Applications}{Miscellaneous}
%A category including the fourth, optional field follows...
\category{D.2.8}{Software Engineering}{Metrics}[complexity measures, performance measures]

\terms{Delphi theory}

\keywords{ACM proceedings, \LaTeX, text tagging}

\section{Introduction}

Past studies have shown that the World Wide Web is organized into web
communities, and that semantically similar sites can be identified
based on their link structures.

\textbf{[TODO: Mike, have a look at this]} -- We hypothesize that a
similar organizational structure is present in the blogosphere: that
blogs are more likely to link to blogs of a similar topic. As such,
algorithms that have been shown to work on the web in general should
also work, but we also develop a novel algorithm that is more suited
to the link patterns observed in blogs.

\section{Related Work}

Gibson, Kleinberg and Raghavan~\cite{gibson1998iwc} studied the link
structure of Internet sites and how they can be used to infer the
existence of communities of related sites.

Dean and Henzinger~\cite{dean1999frp} described two algorithms (one
derived from HITS) that use link information to identify related web
pages.

\section{Algorithms}
\subsection{Secret Sauce}
This algorithm is based on Friend-of-a-Friend (FOAF). The rationale is
that a blog is likely to cite other blogs (that we call the \emph{D1}
set, for distance=1) that share at least a single topic of common
interest. The blogs that these blogs cite (\emph{D2}) should thus also
include those on that shared topic, and the more paths there are to a
\emph{D2} blog (through any blog in \emph{D1}), the more likely this
blog is to be relevant, since more of the source blog's friend agrees
on it.


\subsection{HITS}

\subsection{HITS with Co-citation and Co-reference}

This uses the standard HITS implementation, but the initial subgraph
is enriched with sites that co-refer the same targets and sites that
are co-cited (a parent site links to both the source site and this
site)

\section{Implementation}
\subsection{Graph Framework}

We build our graph framework on top of the open-source Java Unified
Network/Graph Framework, adding


\subsection{Web Application}
We have written a web interface 

%\section{Results}
%\section[Discussion}
\section{Conclusions}

We have developed an algorithm for ranking blogs in the link
neighborhood of a query blog such that the resulting ordering
approximates their semantic similarity to the query.

\section{Future Work}
We plan to do a double-blind user study, aimed at answering the
following questions:
\begin{enumerate}
\item How well does link structure predict semantic structure,
  regardless of the specific algorithm
\item How well does our algorithm fare against similar algorithms,
  e.g. those derived from HITS

\end{enumerate}

%ACKNOWLEDGMENTS are optional
\section{Acknowledgments}
This section is optional; it is a location for you
to acknowledge grants, funding, editing assistance and
what have you.  In the present case, for example, the
authors would like to thank Gerald Murray of ACM for
his help in codifying this \textit{Author's Guide}
and the \textbf{.cls} and \textbf{.tex} files that it describes.

%
% The following two commands are all you need in the
% initial runs of your .tex file to
% produce the bibliography for the citations in your paper.
\bibliographystyle{abbrv}
\bibliography{web-mining}
% You must have a proper ".bib" file
%  and remember to run:
% latex bibtex latex latex
% to resolve all references
%
% ACM needs 'a single self-contained file'!
%

\subsection{References}
Generated by bibtex from your ~.bib file.  Run latex,
then bibtex, then latex twice (to resolve references)
to create the ~.bbl file.  Insert that ~.bbl file into
the .tex source file and comment out
the command \texttt{{\char'134}thebibliography}.
% This next section command marks the start of
% Appendix B, and does not continue the present hierarchy
\section{More Help for the Hardy}
The sig-alternate.cls file itself is chock-full of succinct
and helpful comments.  If you consider yourself a moderately
experienced to expert user of \LaTeX, you may find reading
it useful but please remember not to change it.
%\balancecolumns % GM MARCH 2007
% That's all folks!
\end{document}
